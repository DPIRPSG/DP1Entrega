% !TeX spellcheck = es_ES
\documentclass{scrartcl}
%\documentclass{book}
\usepackage[spanish]{babel}
\usepackage[utf8]{inputenc}
\usepackage{enumerate}
\usepackage{graphicx}
\usepackage{color}
\usepackage{float}
\usepackage{multicol}
\usepackage{listings}%codigo de programación
\usepackage{lscape}
%\usepackage{fancyhdr}
\usepackage{tabularx}
\usepackage[hidelinks,colorlinks=true, linkcolor=black]{hyperref} %hyperref para hacer los links del indice y usar \url{URL} y \href{URL}{text}y hidelinks para que no se rodeen con una caja de color los enlaces. Más información en: http:\\en.wikibooks.org/wiki/LaTeX/Hyperlinks




%\pagestyle{fancy}
%\usepackage{tabulary}
%\usepackage{/home/manolito/IISSI/pruebaLaTex/comfortaa/tex/latex/comfortaa/comfortaa}

%\lfoot[\thepage]{}
%\cfoot[\thepage]{}
%\rfoot[\thepage]{\thepage}
%\lfoot[c1]{GECH}
%\lfoot[e1]{GECH}
%\lhead[\thepage]{}
%\rhead[\thepage]{}
%\renewcommand{\headrulewidth}{0.0pt}
%\renewcommand{\footrulewidth}{0.2pt}
\hyphenation{se-ma-nal per-so-na-les ope-ra-tivo mo-ni-tor mues-tra pide correc-ta-men-te}



%\renewcommand{\baselinestretch}{1.5} %interlineado


\begin{document}
\section{Implementación de un menú distinto a Jmenu}
Decidimos reemplazar el menú dado en la asignatura (Jmenu) por mnMenu (\href{https://github.com/manusa/mnmenu}{GitHub}) debido a la simplicidad, la similitud con el menú original, el ejemplo \href{http://www.marcnuri.com/en/content/mnmenu-dropdown-jquery-menu}{online} y a la documentación y ejemplos descargables desde \href{https://github.com/manusa/mnmenu}{GitHub}.
\\
\\
Su implementación consistió en la inclusión del archivo \href{https://github.com/manusa/mnmenu/blob/master/src/jquery.mnmenu.js}{jquery.mnmenu.js} en la carpeta de nuestro proyecto denominada \href{https://github.com/DPIRPSG/DP1Entrega/tree/master/L01Workspace/Getting-Started-Menu/src/main/webapp/scripts}{/src/main/webapp/scripts}. Posteriormente fue necesario cambiar la llamada que realiza \href{https://github.com/DPIRPSG/DP1Entrega/blob/master/L01Workspace/Getting-Started-Menu/src/main/webapp/views/master-page/layout.jsp}{layout.jsp} (localizado en \href{https://github.com/DPIRPSG/DP1Entrega/blob/master/L01Workspace/Getting-Started-Menu/src/main/webapp/views/master-page/}{webapp/views/master-page}) por la dirección del archivo añadido anteriormente (\href{https://github.com/DPIRPSG/DP1Entrega/tree/master/L01Workspace/Getting-Started-Menu/src/main/webapp/scripts/jquery.mnmenu.js}{jquery.mnmenu.js}).
En ese mismo archivo modificamos el estilo css que solitica jmenu por uno de los que ofrecía mnMenu denominado \href{https://github.com/DPIRPSG/DP1Entrega/blob/Item5-6/L01Workspace/Getting-Started-Menu/src/main/webapp/styles/mnmenu.css}{mnmenu.css} y el script que ejecuta para cargar anteriormente por este:
\begin{lstlisting}[frame=single]
<script>
	$(document).ready(function() {
		$('#bluemenu').mnmenu();
	})
</script>
\end{lstlisting}
Se puede apreciar que al menú se le llama como \textit{bluemenu} y no con \textit{mnmenu}. Esto se debe a que usamos ese estilo visual para el menú por lo que es la manera más simple de ejecutarlo.
\\
\\
Tras esto, debemos acceder al archivo \href{https://github.com/DPIRPSG/DP1Entrega/blob/master/L01Workspace/Getting-Started-Menu/src/main/webapp/views/master-page/header.jsp}{headert.jsp} y modificar los ids y las clases correspondientes a cada sección del menú con el fin de que coincidan con las del estilo css para conservar el estilo original. Ejemplo:
\begin{lstlisting}[frame=single]
<ul id="bluemenu" class="bluemenu"> 
	<li class="middle first level-0"> A </a>
	<ul class="level-1">
		<li class="first level-1"> Sub A - 1 </li>
		<li class="last level-1"> Sub A - 2 </li>					
	</ul>
	</li>
</ul>
\end{lstlisting}



A Pesar de esto, tuvimos que realizar algunas pequeñas modificaciones para facilitar el estilo visual.
\end{document}